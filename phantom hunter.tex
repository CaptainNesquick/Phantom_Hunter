% !TEX TS-program = pdflatex
% !TEX encoding = UTF-8 Unicode

% This is a simple template for a LaTeX document using the "article" class.
% See "book", "report", "letter" for other types of document.

\documentclass[10pt]{article} % use larger type; default would be 10pt

\usepackage[utf8]{inputenc} % set input encoding (not needed with XeLaTeX)

%%% Examples of Article customizations
% These packages are optional, depending whether you want the features they provide.
% See the LaTeX Companion or other references for full information.

%%% PAGE DIMENSIONS
\usepackage{geometry} % to change the page dimensions
\geometry{a4paper} % or letterpaper (US) or a5paper or....
% \geometry{margin=2in} % for example, change the margins to 2 inches all round
% \geometry{landscape} % set up the page for landscape
%   read geometry.pdf for detailed page layout information

\usepackage{graphicx} % support the \includegraphics command and options

% \usepackage[parfill]{parskip} % Activate to begin paragraphs with an empty line rather than an indent

%%% PACKAGES
\usepackage{booktabs} % for much better looking tables
\usepackage{array} % for better arrays (eg matrices) in maths
\usepackage{paralist} % very flexible & customisable lists (eg. enumerate/itemize, etc.)
\usepackage{verbatim} % adds environment for commenting out blocks of text & for better verbatim
\usepackage{subfig} % make it possible to include more than one captioned figure/table in a single float
% These packages are all incorporated in the memoir class to one degree or another...

%%% HEADERS & FOOTERS
\usepackage{fancyhdr} % This should be set AFTER setting up the page geometry
\pagestyle{fancy} % options: empty , plain , fancy
\renewcommand{\headrulewidth}{0pt} % customise the layout...
\lhead{}\chead{}\rhead{}
\lfoot{}\cfoot{\thepage}\rfoot{}

%%% SECTION TITLE APPEARANCE
\usepackage{sectsty}
\allsectionsfont{\sffamily\mdseries\upshape} % (See the fntguide.pdf for font help)
% (This matches ConTeXt defaults)

%%% ToC (table of contents) APPEARANCE
\usepackage[nottoc,notlof,notlot]{tocbibind} % Put the bibliography in the ToC
\usepackage[titles,subfigure]{tocloft} % Alter the style of the Table of Contents
\renewcommand{\cftsecfont}{\rmfamily\mdseries\upshape}
\renewcommand{\cftsecpagefont}{\rmfamily\mdseries\upshape} % No bold!

%%% END Article customizations

%%% The "real" document content comes below...

\title{\Huge Phantom hunter}
\author{Projet sambudjai}
%\date{} % Activate to display a given date or no date (if empty),
         % otherwise the current date is printed 
\renewcommand*\contentsname{Table des matières}
\pagestyle{fancy}

\renewcommand{\headrulewidth}{1pt}
\fancyhead[L]{Phantom Hunter}
\fancyhead[C]{\leftmark}
\fancyhead[R]{Projet Sanbudjai}

\renewcommand{\footrulewidth}{1pt}
\fancyfoot[L]{Sup D1}
\fancyfoot[R]{Epita 2021}

\begin{document}

\maketitle

\newpage

\tableofcontents

\newpage

\section{Introduction}


Ce cahier des charges a pour but de définir les caractéristiques de notre projet ainsi que les différents aspects techniques (Intelligence Artificielle, Réseau) que nous allons y intégrer. Ce projet informatique se déroulera sur six mois et sera ponctue de soutenances.\\

Nous avons décidé de réaliser un jeu de chasse aux fantômes en vue subjective focaliser sur le 
 Multi-joueurs. Dans ce jeu il y aura plusieurs modes de jeu. Pour cela nous avons deux modes de jeu : un mode survie et un mode contre la montre.Il y aura aussi la possibilité de jouer contre l'intelligence artificielle en solo. \\

Dans la suite du cahier des charges vous trouverez plus de détails sur le projet mais aussi sur le budget et la répartition des tâches.
Notre but dans ce projet de second semestre est de vous livrer un jeu offrant la meilleure expérience possible, tout en nous permettant de développer nos compétences en programmation et en gestion de projet.

\newpage


\section{Projet}



\subsection{Présentation}

Notre jeu sera de type chasse aux fantômes c'est a dire qu'il y aura deux équipe: fantômes et chasseurs.
Etant donné que nous aimons tous, les jeux de type multijoueur et les shooters, nous  nous somme mis d'accord pour la création de ce jeu.
Le fantôme et les chasseur s'affronteront dans un environnement de style "Manoir Hanté". Les chasseurs seront armés de lampes torches et le fantôme devra se rapprocher d'un chasseur pour l'éliminer. Les chasseurs assommés pourront être réanimés par d'autres chasseurs.
Ils seront tous armés d'armes spéciales comme une tourelle qui éclaire une zone dans un rayon proche, cet item pourra être utilisé par les chasseur, et le fantôme pourra avoir, entre autres, un totem, destructible par les chasseurs.      
Si tous les chasseurs sont KO alors le fantôme gagne, cependant d'autres modes de jeu pourrait être ajoutés, comme un mode contre-la-montre durant lequel le fantôme devra survivre pendant un temps donné à un assaut des chasseurs.



\subsection{Interêts\\}



\subsubsection{Aymard Théo}

j' ai déja travailler sur unity pour créer un niveau de jeu vidéo, par contre j'ai été très aider du coup il y a beaucoup de chose que je ne connait pas mais je suis content de pouvoir apprendre avec ce type de projet, cela m'apportera des connaissances qui me seront utile pour plus tard, comme une facilité avec des logiciel similaire, la gestion de projet future et d'équipe.Travailler en équipe me parrait attrayant, j'aime beaucoup le fait de pouvoir partager les opignons, les idées...        
   
\subsubsection{Beghdadi Mohamed}

Depuis que je suis tout petit petit j'ai toujours été fasciné par les dessins animés et les jeux vidéo. De ce fait j'ai toujours voulu travailler dans l'informatique. De plus  mon père travaille également  dans ce secteur en tant qu'ingénieur et plus précisément dans la gestion de base de données et c’est vrai que j’ai toujours trouve cela impressionnant. Ainsi j’ai décidé d’en faire mon métier et l’EPITA est pour moi l’occasion de réaliser un rêve d’enfant. De plus c’est la deuxième fois que je réalise ce projet, je trouve ça toujours aussi intéressant et enrichissant.

\subsubsection{Courtois Pierre}

J'ai déjà eu l'occasion de travailler sur des projets techniques en tant que chef de groupe par le passé, mais jamais sur des projets informatiques. Ce projet sera l'occasion pour moi de développer mes capacités en tant que développeur, animateur et sound designer. Je suis aussi très intéressé par le milieu du jeu vidéo, et ce projet me permettra de découvrir les coulisses de ce milieu.

\subsubsection{Rancoule Thomas}



\section{Etat de l'art}



\subsection{Inspirations}



\subsection{Dissemblances}



\section{Découpage du projet}



\subsection{Unity 3D}

Unity3D est un moteur de jeu. Il permet de créer des jeux sur plusieurs types de plateformes comme PC, Mac, consoles, et téléphones. Il nous est très pratique car il permet de programmer en C\#. Unity3D est un logiciel permettant de construire des décors qui correspondent à nos envies. Les possibilités que nous offre Unity3D au niveau graphique sont parfaitement adaptées à notre jeu, et nous permettront alors de créer les maps de notre jeu. Le point le plus important d’Unity est que grâce à lui nous pourrons réaliser notre jeu en vue subjective facilement.

\subsection{Gameplay}



\subsection{Multijoueur}

La partie multijoueur est une parite très importante de notre projet, vu que notre sera fait pour s'amuser entre amis. Sachant que nous n'avons pas un jeu basée sur l'histoire, la partie multijoueur devra contribuer à ce que chaque personne puisse jouer librement et n'ai pas de problème, les joueur seront disperssé dans une équipe de phantom et une équipe de chasseurcependant si il n'y a pas assez de joueur des IA seront répartient dans les équipes, du coup les personne voulant jouer à deux pourront et seront rejoins de bot.    

\subsection{Intelligence Artificielle}

L'intelligence artificielle correspond à une discipline scientifique qui permet au travers d’algorithmes de simuler l'intelligence humaine. Dans notre projet, l'intelligence artificielle aura pour but de rendre "intelligent"les ennemis afin que ceux-ci puissent combattre de manière efficace. De plus, nous ferons aussi un algorithme de Pathfinding.Celui-ci premettra aux ennemis se dirigeant vers notre fantômes d'être capables d’éviter les obstacles et de trouver le chemin le plus rapide pour se rendre jusqu'à lui. L'intelligence artificielle sera présente aussi dans le mode multi-joueurs si le joueur le désir.

\subsection{Graphisme}



\subsection{Son}

Le son devra à la fois permettre au jeu d'être plus lisible, en comprenant à l'oreille les actions effectuées(objets utilisés, pas des chasseurs, interactions avec l'environnement...), et rendre l'atmosphère pesante pour exacerber le côté paranormal du jeu. Dans le menu, la musique devra être discrète, pour ne pas devenir désagréable rapidement, étant donné que cet écran sera fréquemment visité par le joueur. Durant une partie, la musique devra rendre l'atmosphère pesante, tout en restant suffisamment sobre pour ne pas sortir les joueurs de l'ambiance. Les crédits posséderont aussi leur propre thème.
Le son devra à la fois permettre au jeu d'être plus lisible, en comprenant à l'oreille les actions effectuées (objets utilisés, pas des chasseurs, interactions avec l'environnement...), et rendre l'atmosphère pesante pour exacerber le côté paranormal du jeu. Dans le menu, la musique devra être discrète, pour ne pas devenir désagréable rapidement, étant donné que cet écran sera fréquemment visité par le joueur. La musique d'une partie sera choisie aléatoirement parmi une liste de musiques. Durant une partie, la musique devra rendre l'atmosphère pesante, tout en restant suffisamment sobre pour ne pas sortir les joueurs de l'ambiance. Les crédits posséderont aussi leur propre thème.

\subsection{Site}



\subsection{Cinématiques}

Les cinématiques seront faîtes sur Source FilmMaker, un logiciel d'animation développé par Valve et disponible gratuitement. Les cinématiques seront placées en début et en fin de partie, avec une cinématique de début de partie pour le fantôme, une cinématique de début de partie pour les chasseurs, une cinématique en cas de victoire du fantôme, et une cinématique en cas de victoire des chasseurs.Ces cinématiques seront un plus, qui permettront de dynamiser le début et la fin des parties. Elles permettent de mettre en valeur chaque camp et fait fonctionner notre imagination .Mais aussi rajoute un élément essentiel aux jeux d'aujourd'hui. Si les joueurs se lassent des cinématiques, il sera possible de les passer si tous les joueurs appuie sur la touche indiqué.    

\section{Planning}



\subsection{Répartition des tâches}



\subsection{Planning de soutenance}



\section{Technologies et coûts}



\subsection{Matériel}



\subsection{Logiciels}



\section{Conclusion}






\end{document}
